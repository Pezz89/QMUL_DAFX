\documentclass[titlepage]{scrartcl}
\usepackage{enumitem}
\usepackage[british]{babel}
\usepackage[style=apa, backend=biber]{biblatex}
\DeclareLanguageMapping{british}{british-apa}
\usepackage{url}
\usepackage{float}
\usepackage[labelformat=empty]{caption}
\restylefloat{table}
\usepackage{perpage}
\MakePerPage{footnote}
\usepackage{abstract}
\usepackage{graphicx}
% Create hyperlinks in bibliography
\usepackage{hyperref}
\usepackage{amsmath}

\usepackage[T1]{fontenc}
\usepackage[utf8]{inputenc}
\usepackage{blindtext}
\setkomafont{disposition}{\normalfont\bfseries}

\graphicspath{
    {./resources/},
}
\addbibresource{~/Documents/library.bib}

\DeclareSourcemap{
    \maps{
        \map{ % Replaces '{\_}', '{_}' or '\_' with just '_'
            \step[fieldsource=url,
                  match=\regexp{\{\\\_\}|\{\_\}|\\\_},
                  replace=\regexp{\_}]
        }
        \map{ % Replaces '{'$\sim$'}', '$\sim$' or '{~}' with just '~'
            \step[fieldsource=url,
                  match=\regexp{\{\$\\sim\$\}|\{\~\}|\$\\sim\$},
                  replace=\regexp{\~}]
        }
    }
}

\newsavebox{\abstractbox}
\renewenvironment{abstract}
  {\begin{lrbox}{0}\begin{minipage}{\textwidth}
   \begin{center}\normalfont\sectfont\abstractname\end{center}\quotation}
  {\endquotation\end{minipage}\end{lrbox}%
   \global\setbox\abstractbox=\box0 }

\usepackage{etoolbox}
\makeatletter
\expandafter\patchcmd\csname\string\maketitle\endcsname
  {\vskip\z@\@plus3fill}
  {\vskip\z@\@plus2fill\box\abstractbox\vskip\z@\@plus1fill}
  {}{}
\makeatother

\DeclareCiteCommand{\citeyearpar}
    {}
    {\mkbibparens{\bibhyperref{\printdate}}}
    {\multicitedelim}
    {}

% MATLAB Code block stuff...
\usepackage{color}
\usepackage{listings}

\definecolor{dkgreen}{rgb}{0,0.6,0}
\definecolor{gray}{rgb}{0.5,0.5,0.5}

\lstset{language=Matlab,
   keywords={break,case,catch,continue,else,elseif,end,for,function,
      global,if,otherwise,persistent,return,switch,try,while},
   basicstyle=\ttfamily,
   keywordstyle=\color{blue},
   commentstyle=\color{gray},
   stringstyle=\color{dkgreen},
   numbers=left,
   numberstyle=\tiny\color{gray},
   stepnumber=1,
   numbersep=10pt,
   backgroundcolor=\color{white},
   tabsize=4,
   showspaces=false,
   showstringspaces=false}

\begin{document}
\title{ECS730P --- Digital Audio Effects}
\subtitle{\LARGE{Assignment 2 Report}}
\author{Sam Perry --- EC16039}

\maketitle

\section{Design Overview}
The plugin demonstrates an implementation of a real-time ``granular shuffle''.
The effects works by randomly selecting small segments of audio from previous
input, and mixing these back in with current input in continuous overlapping
grains. The aim of this effect is to allow the user to build up rich textures
through the build up of previous material in a stochastic fashion. Potential
applications for this effect include creative sound design and electro-acoustic
composition. The plugin also remains relevant in a standard audio production
context, as it's ability to create thick textures in a less traditional manner
may provide an interesting alternative to traditional delay based effects.

\section{Granular Synthesis Implementation}
The following steps are taken in order to realise the output of this effect:
\begin{enumerate}
    \item Audio is recorded into a buffer of set size from input.
    \item Overlapping grains of audio are then read from this buffer randomly
        at the grain size specified. The amount of buffer to use is specified
        by the user to determine the maximum distance in time allowed between
        selected grains and the current point in time.
    \item A Hann window is applied to these grains and mixed by a
        user-definable amount with the original dry signal, to create an effect
        comparable to delay and reverb.
\end{enumerate}

Due to time constraints, the original plan for this effect could not be
implemented in a robust or reliable manner. Therefore the decision was made to
focus on ensuring the granular synthesis engine performed accurately over
implementing further new features (as described in section~\ref{OriginalDesign}). A number of
issues were dealt with to ensure the correct operation of the synthesis engine:
\begin{itemize}
    \item Due to the ability to alter grain size and the length of buffer used
        for grain selection, there is a risk of grains being cut short or
        distorting as these parameters are changed. This can happen when a
        change of grain size alters the variables being used in a currently
        played grain. For this reason, grain playback was implemented in such a
        way that any played grain was isolated with set parameters until it had
        fully finished, avoiding unwanted changes mid-playback.
    \item Buffer playback size was also implemented in a way that would not
        have any affect on grains that had begun before a change, allowing for
        seamless transitions between varying buffer and grain sizes.
    \item There is also the risk of grains of varying sizes going out of sync,
        thus loosing the effect of an evenly overlapped stream of audio. A
        global sample counter was used to address this issue. By triggering
        samples based on the current global grain size, grains remain
        overlapped at all points (except for discontinuities between grains of
        varying sizes, which is unavoidable in this situation).
\end{itemize}

\section{User Interface/Parameter Selection}
Throughout the design of the effect, focus was placed on simplicity and ease of
use for the user. Through prior experience with granular synthesis based
effects, it was noticed that there is a tendency for large numbers of complex
parameters to be used in interfaces. An example of this is IRCAM's CataRT
concatenative synthesis engine, which, although it is an excellent piece of
software capable of creating a wide variety of interesting sounds, requires a
great deal of prior knowledge for a user to utilize it effectively. For this
reason, this product aimed to minimize the number of parameters available to the user in
order to create a product that is intuitive and creates satisfying results
quickly. This was inspired primarily by the Waves ``OneKnob'' series of audio
plugins, which aim to produce one particular effect well for a variety of use
cases.
Parameters such as the window overlap factor have been hidden from the user,
having been set statically to a value of 4 which has been found to provide the
best trade of between low density of grains and the inherent phase issues that
occur with high overlap factors.

\section{Original Design Concept}\label{OriginalDesign}

\section{Evaluation}
The effect's stochastic behaviour in it's selection of grains produces a more
unpredictable output than that of a standard delay. However, the abaility to
control the degree to which grains deviate from the current sample, alongside
the duration of these grains allows for the regulation of this effect.
\section{Problems/Potential improvements}
addition of filters on wet mix
\section{Conclusion}


\end{document}
