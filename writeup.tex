\documentclass[titlepage]{scrartcl}
\usepackage{enumitem}
\usepackage[british]{babel}
\usepackage[style=apa, backend=biber]{biblatex}
\DeclareLanguageMapping{british}{british-apa}
\usepackage{url}
\usepackage{float}
\usepackage[labelformat=empty]{caption}
\restylefloat{table}
\usepackage{perpage}
\MakePerPage{footnote}
\usepackage{abstract}
\usepackage{graphicx}
% Create hyperlinks in bibliography
\usepackage{hyperref}
\usepackage{amsmath}

\usepackage[T1]{fontenc}
\usepackage[utf8]{inputenc}
\usepackage{blindtext}
\setkomafont{disposition}{\normalfont\bfseries}

\graphicspath{
    {./resources/},
}
\addbibresource{~/Documents/library.bib}

\DeclareSourcemap{
    \maps{
        \map{ % Replaces '{\_}', '{_}' or '\_' with just '_'
            \step[fieldsource=url,
                  match=\regexp{\{\\\_\}|\{\_\}|\\\_},
                  replace=\regexp{\_}]
        }
        \map{ % Replaces '{'$\sim$'}', '$\sim$' or '{~}' with just '~'
            \step[fieldsource=url,
                  match=\regexp{\{\$\\sim\$\}|\{\~\}|\$\\sim\$},
                  replace=\regexp{\~}]
        }
    }
}

\newsavebox{\abstractbox}
\renewenvironment{abstract}
  {\begin{lrbox}{0}\begin{minipage}{\textwidth}
   \begin{center}\normalfont\sectfont\abstractname\end{center}\quotation}
  {\endquotation\end{minipage}\end{lrbox}%
   \global\setbox\abstractbox=\box0 }

\usepackage{etoolbox}
\makeatletter
\expandafter\patchcmd\csname\string\maketitle\endcsname
  {\vskip\z@\@plus3fill}
  {\vskip\z@\@plus2fill\box\abstractbox\vskip\z@\@plus1fill}
  {}{}
\makeatother

\DeclareCiteCommand{\citeyearpar}
    {}
    {\mkbibparens{\bibhyperref{\printdate}}}
    {\multicitedelim}
    {}

% MATLAB Code block stuff...
\usepackage{color}
\usepackage{listings}

\definecolor{dkgreen}{rgb}{0,0.6,0}
\definecolor{gray}{rgb}{0.5,0.5,0.5}

\lstset{language=Matlab,
   keywords={break,case,catch,continue,else,elseif,end,for,function,
      global,if,otherwise,persistent,return,switch,try,while},
   basicstyle=\ttfamily,
   keywordstyle=\color{blue},
   commentstyle=\color{gray},
   stringstyle=\color{dkgreen},
   numbers=left,
   numberstyle=\tiny\color{gray},
   stepnumber=1,
   numbersep=10pt,
   backgroundcolor=\color{white},
   tabsize=4,
   showspaces=false,
   showstringspaces=false}

\begin{document}
\title{ECS730P --- Digital Audio Effects}
\subtitle{\LARGE{Assignment 2 Report}}
\author{Sam Perry --- EC16039}

\maketitle

\section{Design Overview}
The plugin demonstrates an implementation of a real-time ``granular shuffle''.
The effect works by randomly selecting small segments of audio from previous
input, and mixing these back in with current input in continuous overlapping
grains. The aim of this effect is to allow the user to generate rich textures
through the build up of previous material in a stochastic fashion. Potential
applications for this effect include creative sound design and electro-acoustic
composition. The plugin also remains relevant in a standard audio production
context, as it's ability to create thick textures in a less traditional manner
may provide an interesting alternative to standard delay based effects.

\section{Granular Synthesis Implementation}
The following steps are taken in order to realise the output of this effect:
\begin{enumerate}
    \item Audio is recorded into a buffer of set size from input.
    \item Overlapping grains of audio are then read from this buffer randomly
        at the grain size specified. The amount of buffer to use is specified
        by the user to determine the maximum distance in time allowed between
        selected grains and the current point in time.
    \item A Hann window is applied to these grains and mixed by a
        user-definable amount with the original dry signal. This creates an effect
        comparable to delay and reverb.
\end{enumerate}

Due to time constraints, the original plan for this effect could not be
implemented in a robust or reliable manner. Therefore the decision was made to
focus on ensuring the core granular synthesis engine performed accurately over
implementing further new features (as described in
section~\ref{OriginalDesign}). A number of issues were dealt with to ensure the
correct operation of the synthesis engine:

\begin{itemize}
    \item Due to the ability to alter grain size and the length of buffer used
        for grain selection, there is a risk of grains being cut short or
        distorting as these parameters are changed. This can happen when a
        change of grain size alters the variables being used in a currently
        played grain. For this reason, grain playback was implemented in such a
        way that any played grain was isolated with set parameters until it had
        fully finished, avoiding unwanted changes mid-playback.
    \item Buffer playback size was also implemented in a way that would not
        have any effect on grains that had begun before a change, allowing for
        seamless transitions between varying buffer and grain sizes.
    \item There is also the risk of grains of varying sizes going out of sync,
        thus loosing the effect of an evenly overlapped stream of audio. A
        global sample counter was used to address this issue. By triggering
        samples based on the current global grain size, grains remain
        overlapped at all points (with the exception of possible
        discontinuities between grains of varying sizes, which is unavoidable
        in this situation).
\end{itemize}

\section{User Interface/Parameter Selection}
Throughout the design of the effect, focus was placed on simplicity and ease of
use for the user. Through prior experience with granular synthesis based
effects, it was noticed that there is a tendency for large numbers of complex
parameters to be used in granular synthesis interfaces. An example of this is IRCAM's CataRT
concatenative synthesis engine~\parencite{Schwarz2006a}, which, although it is
an excellent piece of software capable of creating a wide variety of
interesting sounds, requires a great deal of prior knowledge for a user to
utilize it effectively. For this reason, this product aimed to minimize the
number of parameters available to the user in order to create a product that is
intuitive and creates satisfying results quickly. This was inspired primarily
by the Waves ``OneKnob'' series of audio plugins~\parencite{Waves2017} which
aim to produce one particular effect well for a variety of use cases.\\
Parameters such as the window overlap factor have been hidden from the user,
having been set statically to a value of 4 which has been found to provide the
best trade of between low density of grains and the inherent phase issues that
occur with high overlap factors.

\section{Original Design Concept}\label{OriginalDesign}
The original concept for this project builds on methods described previously.
The complete design included one significant improvement to the selection of
grains: The original idea focused on the addition of a prior spectral flux
analysis of the signal~\parencite[p.44]{Lerch2012} in order to separate
transient and stable segments of audio (An example spectral flux analysis on a
piano sample is shown in figure~\ref{PianoFlux}. The distinction between note
onsets and steady parts can be clearly seen in the peaks of the analysis). It
has been observed that percussive and non-stationary audio benefits from a
short grain window in granular synthesis, due to the quick changes in it's
temporal characteristics.  Conversely, stable audio requires a longer grain
size to allow for smoother transitions between grains. For this reason spectral
flux analysis would have been used to select between two window sizes based on
the classification of the signal using a hysteresis. Unfortunately, due to time
constraints, the implementation of this feature was not possible.

\section{Evaluation}
A number of examples have been generated to demonstrate the performance of the
effect on a variety of sources. One example that is of particular interest is the
spoken word loop. This effect allowed for the creation of a simplistic ``cocktail bar''
effect from a relatively small piece of audio.
The use of separate synthesis engines on a per-channel basis has also resulted
in interesting stereo variations. The mixture of this with a centred dry
signal creates particularly interesting textures. 
Although the relatively unpredictable selection of output grains tends to
distort rhythmic samples (this is most noticeable in the drum loop example), it
can still be used more subtle by blending the effect with the dry signal to
produce interesting results, whilst maintaining the overall rhythm of the
input.\\

From a technical point of view, the effect performs as expected. Tests using
white noise show that grains are generated with the expected temporal shape.
This can be observed on a basic level by simply inspecting the shape of the
waveform of output. This can be seen in figure~\ref{Temporal} where initially
there is clear overlap between multiple grains, and in the later half
individual grains can be seen as overlapping with silence, showing there
temporal shape more clearly.
Particular care has been taken in maintaining sample accuracy in the
overlapping of consecutive grains, and the application of the effect over long
periods does not result in any noticeable synchronisation issues. This may
require further tests for total certainty. 

\section{Problems/Potential improvements}
\begin{itemize}
    \item There are currently no known bugs in the project, due to a reasonable
        amount of time spent testing code to ensure correct operation using
        tools such as LLDB and MrsWatson for debugging.
    \item It has been suggested that the addition of filters on the wet mix
        might allow for greater control over the blend between the effect and
        the dry signal. This would be a simple addition to the plugin, however
        this would require further parametrisation, which would require some
        thought to ensure the intuitive feel of the plugin is maintained.
    \item A further area that could be explored is the concept of spectral
        granularisation, which applied a similar technique of randomised
        granular shuffling of bins in the frequency domain.
\end{itemize}

\section{Conclusion}
Overall this project has demonstrated the potential for granular synthesis in
the context of audio effects plugins; The effect's stochastic behaviour in it's
selection of grains produces a more unpredictable output than that of a
standard delay. However, the ability to control the degree to which grains
deviate from the current sample, alongside the duration of these grains allows
for the regulation of this effect in a similar way to methods used for control
of reverb time in parametric reverb algorithms. A variety of examples have shown it's
versatility across different inputs and possible further areas of exploration
have been discussed briefly.

\section{Figures}
\begin{figure}[H]
    \caption{Waveform of effect applied to speech}
    \makebox[\textwidth]{\includegraphics[width=1.1\textwidth]{Temporal}}
    \label{Temporal}
\end{figure}
\begin{figure}[H]
    \caption{Spectral-flux analysis of a piano performance}
    \makebox[\textwidth]{\includegraphics[width=1.1\textwidth]{PianoFlux}}
    \label{PianoFlux}
\end{figure}
\pagebreak
\printbibliography

\end{document}
